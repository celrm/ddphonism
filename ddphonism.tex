\documentclass{article}

\usepackage{hyperref}
\usepackage{multicol}
\usepackage{listings}

\usepackage{ddphonism}

\title{The \textsf{ddphonism} package\footnote{This
		document corresponds to \textsf{ddphonism} v0.1, dated 2019/08/10.}}
\author{Celia Rubio Madrigal\footnote{Email: \href{mailto:celrubio@ucm.es}{\texttt{celrubio@ucm.es}}}}
\date{August 10, 2019}


\begin{document}
	
	\maketitle
	
	\begin{abstract}
		This is a music-related package which is focused on notation from the Twelve-Tone System, also called Dodecaphonism. It provides \LaTeX{} algorithms that produce typical T.T.S. notation based off a musical series, or row sequence, of variable length.
		
		\begin{center}
			\textbf{Keywords}
		\end{center}
	
		\textit{twelve tone system, dodecaphonism, music, mathematics, matrix, row, series, diagram, clock diagram, notation, algorithm, schoenberg, contemporary music, 20th century}
	\end{abstract}

	\tableofcontents
	
	\section{Introduction}
	There are hundreds of music tools and software online which are able to produce different music notations. However, I have never seen a \LaTeX{} tool that can do the same. This package is not only about notation, but it also calculates mathematically how this notation should work.
	
	It is said that a twelve-tone matrix is the only thing a twelve-tone composer should need, because it provides the whole serial spectrum with which they may work. I wanted \LaTeX{} users to be able to generate them automatically.
	
	But I also think that a twelve-tone matrix is not enough, that there exist several other notations with which they may understand their series and its potential. These are the diagrams that can be obtained with this package.
	
	\begin{multicols}{2}
		\dmatrix{0,1,2,3,4}
		
		\ddihedral{4,5,7,1,6,3,8,2,11,0,9,10}
		
		\ddiagram{4,5,7,1,6,3,8,2,11,0,9,10}
		
		\darrows{4,5,7,1,6,3,8,2,11,0,9,10}
		
		\bigskip
		\drow{4,3,2,1,0}
	\end{multicols}
	
	\section{Using the \textsf{ddphonism} package}
	These are the commands provided by \textsf{ddphonism}. The main parameter in every command is the row sequence.
	
	\newcommand{\I}[1]{\item[\texttt{$\backslash$#1}]\quad}
	\begin{itemize}
		\I{dmatrix} produces a twelve-tone matrix of arbitrary length, as shown in \href{https:matrices.netlify.com}{this website}. For example, \verb|\dmatrix{0,2,1,4,3,6,5}| produces the matrix \dmatrix{0,2,1,4,3,6,5}

		The optional parameter \textsf{sep} scales the matrix. The optional parameters \textsf{vsep, hsep} scales the matrix vertically and horizontally.
		
		The optional parameter \textsf{lines} draws lines between rows and columns. The optional parameters \textsf{outside lines, inside lines} only draws the outside or inside lines. The optional parameters \textsf{vlines, hlines} only draws the vertical or horizontal lines.
		
		\verb|\dmatrix[lines,sep=0.75]{0,2,1,4,3,6,5}| produces the matrix\\ 
		\dmatrix[lines,sep=0.75]{0,2,1,4,3,6,5}
		
		The optional parameter \textsf{no tikz} deletes the tikz environment and lets the user write it instead.
		
		\I{ddiagram} produces a twelve tone clock diagram of arbitrary length, as shown in \href{https:diagramas.netlify.com}{this website}. For example, \verb|\ddiagram{0,2,1,4,3,6,5}| produces the diagram\\ 
		\ddiagram{0,2,1,4,3,6,5}
		
		The optional parameter \textsf{name} lets the user write a name at the center of the diagram.
		
		The optional parameter \textsf{up} lets the user choose which number is up north. The default value is the first number in the row.
		
		\verb|\ddiagram[name=P, up=5]{0,2,1,4,3,6,5}| produces the diagram\\
		\ddiagram[name=P, up=5]{0,2,1,4,3,6,5}
		
		The optional parameter \textsf{no tikz} deletes the tikz environment and lets the user write it instead. The option \textsf{up} does not work anymore and the up position becomes 0. It is recommended that the user passes the option \textsf{ddiagram} to the environment:
		\begin{verbatim}
			\begin{tikzpicture}[ddiagram]
			\ddiagram[no tikz]{0,2,1,4,3,6,5}
			\end{tikzpicture}
		\end{verbatim} produces the same diagram as  \verb|\ddiagram{0,2,1,4,3,6,5}|.
		
		\I{ddihedral} produces a dihedral representation of a series of arbitrary length. For example, \verb|\ddihedral{0,2,1,4,3,6,5}| produces the diagram\\
		\ddihedral{0,2,1,4,3,6,5}
		
		The optional parameters \textsf{t, s, c, v} let the user apply transformations to the diagram: \textit{transposition}, \textit{inversion}, \textit{cyclic shift} and \textit{retrograde}, in that order.
		
		\verb|\ddihedral[s=1, c=4]{0,2,1,4,3,6,5}| produces the diagram\\
		\ddihedral[s=1, c=4]{0,2,1,4,3,6,5}
		
		The optional parameter \textsf{no tikz} deletes the tikz environment and lets the user write it instead. It is recommended that the user passes the option \textsf{ddihedral} to the environment:
		\begin{verbatim}
		\begin{tikzpicture}[ddihedral]
		\ddihedral[no tikz]{0,2,1,4,3,6,5}
		\end{tikzpicture}
		\end{verbatim} produces the same diagram as  \verb|\ddihedral{0,2,1,4,3,6,5}|.		
		
		\I{darrows} produces the arrows from the \verb|\ddihedral| diagram. For example,
 		\verb|\darrows{0,2,1,4,3,6,5}| produces the arrows\\
 		\darrows{0,2,1,4,3,6,5}
 		
 		The optional parameter \textsf{no tikz} deletes the tikz environment and lets the user write it instead.
		
		\I{drow} produces a twelve-tone row sequence as a permutation in its matrix form. For example, \verb|\drow{0,2,1,4,3,6,5}| produces the row
		
		\drow{0,2,1,4,3,6,5}
		
		The optional parameter \textsf{sep} lets the user choose the column separation.
		
		\verb|\drow[sep=10pt]{0,2,1,4,3,6,5}| produces the row
		
		\drow[sep=10pt]{0,2,1,4,3,6,5}
		
	\end{itemize}
	
	\section{The package code}	
	\lstset{
		language=[Latex]Tex,
		%
		basicstyle=\footnotesize\sffamily,
		keywordstyle=\footnotesize\sffamily,
		identifierstyle=\footnotesize\sffamily,
		commentstyle=\footnotesize\sffamily,
		stringstyle=\footnotesize\sffamily,
		escapechar=¬,
		%		
		numberstyle=\footnotesize\sffamily,%\ttfamily\tiny\color[gray]{0.3},
		numbers=left,
		stepnumber=2,
		numbersep=15pt,
		%
		columns=flexible,
		showstringspaces=false,
		tabsize=4,
	}
	\lstinputlisting{ddphonism.sty}
	
\end{document}
